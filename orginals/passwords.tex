\documentclass[11pt,a4paper]{article}
\usepackage[utf8x]{inputenc}
\usepackage{ucs}
\usepackage{amsmath}
\usepackage{amsfonts}
\usepackage{amssymb}

\usepackage{color}
\usepackage{xcolor}
\usepackage{listings}

\usepackage[margin=1in]{geometry}

\usepackage{titling}
\setlength{\droptitle}{-1in}

\usepackage{courier}

\setlength{\parindent}{0.2in}

\title{System Administration Lab 8: Passwords\vspace{-2ex}}
\author{Lisa Goeller}
\date{}

\begin{document}

\maketitle

\emph{``Passwords are like underwear. You shouldn’t leave them out where people can see them. You should change them regularly. And you shouldn’t loan them out to strangers."}
\newline
\null\hfill\hfill\emph{-Anonymous}

\subsection*{Readings}

\begin{enumerate}
\item A brief introduction to what encryption is and the various problems associated with it for most people: http://www.washingtonpost.com/blogs/wonkblog/wp/2013/06/14/nsa-proof-encryption-exists-why-doesnt-anyone-use-it/

\item An overview of /etc/passwd and /etc/shadow: http://tuxthink.blogspot.com/2012/03/look-into-etcpasswdetcshadow-and.html
\item More on /etc/passwd: http://www.linfo.org/etc-passwd.html
\item An explanation of how PGP works: http://www.pgpi.org/doc/pgpintro/
\item An introduction to John the Ripper: http://www.admin-magazine.com/Articles/John-the-Ripper

\end{enumerate}

\subsection*{Introduction}

\indent\indent A weak password can create a hole in an otherwise completely secure system.  All it takes is one user with a simple password and someone who knows their way around unix to compromise your system. As people continue to create more sophisticated password crackers, it's becoming easier and easier for  passwords to be guessed. It is incredibly important to make sure that your users have secure passwords to limit security threats as much as possible.

It may seem hard to believe, but people will still choose weak passwords if given the option. SplashData, a company that manages passwords, posts its most common passwords every year. Take a look at the most common passwords of 2013, according to SplashData:



1. 123456

2. password

3. 12345678

4. qwerty

5. abc123

6. 12356789

7. 111111

8. 1234567

9. iloveyou

10. adobe1234

As you can see, weak passwords are still VERY prevalent on the interwebs. In this lab, we will explore what you can do with passwords and encryption. 

\subsection*{Passwords}

\indent\indent The passwords on your system are stored in the \verb|/etc/passwd| and \verb|/etc/shadow| files. The \verb|/etc/passwd| file is world readable because many commands, like \verb|ls|, rely on it to display file ownerships and similar characteristics. The \verb|/etc/passwd| contains account information and stores passwords as a single `x' character while the \verb|/etc/shadow| file, which only gives root read and write permissions, contains the encrypted passwords and personal information about the user. Both files store this information in a single line of code for each user in their respective files.

User accounts are not only limited to humans. If you take a look at either your \verb|/etc/passwd| or \verb|etc/shadow| files, you'll see that they contain more user accounts than there are human users on your machine. Many commands, like \verb|finger| need to have a user account in order to execute properly. 


\subsection*{John the Ripper}

\indent\indent John the Ripper is a password cracking tool used by some administrators to determine the strength of their users' passwords.  John runs checks of your current passwords against its preloaded list of common passwords to see how secure your system's passwords are.

As you know from the reading, John can run in regular, single, wordlist, and incremental modes. Just to elaborate, single mode uses a dictionary created out of a user's gecos fields, entries in the \verb|/etc/passwd| file that contain general information about users like their real name and contact informatin. In incremental mode, John will try all possible variations of ASCII characters for passwords up to 13 characters long. Just as a warning, John can take up a lot of CPU processing time and it can sometimes take a couple days to crack a password. 

Note: Many have been arrested for using John the Ripper on a network. Make sure before running password crackers to always let the higher ups know if you are not doing it on your own machine. 


Now that you know a bit about John, try it out yourself!

You should see a file called jackson.txt in your home directory. Take a look at it.

\begin{lstlisting}[basicstyle=\ttfamily, backgroundcolor = \color{lightgray}, language = bash, xleftmargin = 0cm, framexleftmargin = 1em]
cat jackson.txt
\end{lstlisting}

Recall from the readings what each of these fields are. What can you tell about mike's account just from looking at this files? Run John against the jackson.txt file.

\begin{lstlisting}[basicstyle=\ttfamily, backgroundcolor = \color{lightgray}, language = bash, xleftmargin = 0cm, framexleftmargin = 1em]
/usr/sbin/john jackson.txt
\end{lstlisting}

John will soon let you know it's completed its job. By default, John runs through single crack mode first, then uses a wordlist with rules, and finally goes for incremental mode. Once cracked, the passwords are stored in the \verb|~john.pot| file. To show them, you'll have to run another command as root.

\begin{lstlisting}[basicstyle=\ttfamily, backgroundcolor = \color{lightgray}, language = bash, xleftmargin = 0cm, framexleftmargin = 1em]
/usr/sbin/john --show jackson.txt
\end{lstlisting}

You can now see mike's password -- it's one of my favorite tunes!




\subsection*{Choosing a Strong Password}

\indent\indent You're probably familiar with the basic requirements of a strong password. It's generally accepted that a password should have a minimum length of eight characters and contain at least one number, one non-alphanumeric character, and one letter. It can be difficult coming up with passwords on your own. Gentoo does offer some tools for random password generation. We're going to look at one of them, \verb|passook|.

You can use \verb|passook| to generate some random passwords. The general syntax for \verb|passook| is as follows: 

\begin{lstlisting}[basicstyle=\ttfamily, backgroundcolor = \color{lightgray}, language = bash, xleftmargin = 0cm, framexleftmargin = 1em]
passook [-n num] [-p plev] [-u usernames]
\end{lstlisting}

where \verb|num| is the number of passwords you want to generate, \verb|plev| is the pronunciation level, on a scale from 1 (all non-alphanumeric characters) to 5 (no non-alphanumeric characters), and \verb|usernames| is the path to the usernames file. If the \verb|usernames| option is included, \verb|passook| will automaticlaly add the passwords it generates to the usernames in the file. 

For funsies, try generating a couple passwords with different pronunciation levels. 

\subsection*{cracklib and pam}

\indent\indent Now that you've seen what strong passwords look like, it's time to ensure that all the passwords on your system will meet your own password requirement, whatever that may be.

\verb|pam|, the \verb|pluggable authentication module|, is used for authentication between different services. \verb|pam| acts as a layer between authentication services and the programs that need authentication before they can proceed. It's used by tons of applicatoins. Run the command below to see the list of applications with \verb|pam| support. 

\begin{lstlisting}[basicstyle=\ttfamily, backgroundcolor = \color{lightgray}, language = bash, xleftmargin = 0cm, framexleftmargin = 1em]
ldd /{,usr/}{bin,sbin}/* | grep -B 5 libpam | grep '^/'
\end{lstlisting}

You can alternatively see the list of packages that depend on \verb|pam|: 

\begin{lstlisting}[basicstyle=\ttfamily, backgroundcolor = \color{lightgray}, language = bash, xleftmargin = 0cm, framexleftmargin = 1em]
equery depends pam
\end{lstlisting}

As you can see, \verb|pam| is a very important part of your system. \verb|pam|'s functions can be roughly divided into four jobs: \verb|authentication, account, session,| and \verb|password.| Authentication makes sure that the user has valid credentials, account checks whether the user has permission to access whatever they're trying to do, session builds up the environment, and password is for updating and modifying a user's account information.  

\verb|cracklib| is an add-on feature that can be used to enhance \verb|pam's authentication| function, forcing users to create stronger passwords. We will explore some of these actions now.


\begin{enumerate}


\item Go to \verb|/etc/pam.d/passwd| and edit it so that it somewhat resembles the code below.

\begin{small}
\begin{lstlisting}[basicstyle=\ttfamily, backgroundcolor = \color{lightgray}, language = bash, xleftmargin = 0cm, framexleftmargin = 1em]
auth     required pam_unix.so shadow nullok
account  required pam_unix.so
password required pam_cracklib.so retry=3 minlen=8 enforce_for_root
password required pam_unix.so md5 use_authtok
session  required pam_unix.so
\end{lstlisting}
\end{small}

These are just some example options. The \verb|retry=3| command gives the user three tries to get a proper password and \verb|minlen=8| requires a password that is at least eight characters long. Normally these options only enforce these requirements for normal users. Adding the \verb|enforce_for_root| option forces the root user to follow the password requirements as everyone else. To see more options, simply type \verb|man pam_cracklib| into the command line.

\item Create a fake user \verb|maninthemirror|.

\begin{lstlisting}[basicstyle=\ttfamily, backgroundcolor = \color{lightgray}, language = bash, xleftmargin = 0cm, framexleftmargin = 1em]
useradd maninthemirror
\end{lstlisting}
Now try giving \verb|maninthemirror| a password.

\begin{lstlisting}[basicstyle=\ttfamily, backgroundcolor = \color{lightgray}, language = bash, xleftmargin = 0cm, framexleftmargin = 1em]
passwd maninthemirror
\end{lstlisting}

You'll notice that if you try to enter a password that doesn't meet the minimum requirements that you described in your \verb|/etc/pam.d/passwd| file then you will be unable to set \verb|maninthemirror's| password. \verb|cracklib| is thus a very powerful tool to secure your system. 

\end{enumerate}

\subsection*{Dictionaries}
\verb|cracklib| comes equipped with a simple dictionary, \verb|cracklib-small|. This is a basic set of words to run your potential passwords against. On the \verb|cracklib| download page, there are plenty of other wordlists that you can download and use to test the strength of your passwords. All of your dictionaries are automatically stored in \verb|/usr/share/dict|, including the \verb|words| dictionary that comes as a default on all Unix systems and is used by word processessors like \verb|vim| for spellchecking. If you add another dictionary, you can have cracklib include this dictionary and all the other dictionaries when checking password strength simply by typing this command:

\begin{lstlisting}[basicstyle=\ttfamily, backgroundcolor = \color{lightgray}, language = bash, xleftmargin = 0cm, framexleftmargin = 1em]
create-cracklib-dict /usr/share/dict/*
\end{lstlisting}

\verb|create-cracklib-dict| takes wordlist files and converts them into \verb|cracklib| dictionaries so that \verb|cracklib| can use them. \verb|cracklib| creates three files in your \verb|/usr/lib| directory: \verb|cracklib_dict.hwm, cracklib_dict.pwd| and \verb|cracklib_dict.pwi|. These are the files that cracklib looks up whenever it's doing a check.  

When you run \verb|create-cracklib-dict| you'll see that two numbers are outputted. The first is the number of words that \verb|cracklib| found in your wordlist. The second is the number of words from that file that it added to its dictionary.


\subsection*{Encrypting Files}

If you want only certain people to have access to your file, you can change the file's permissions so that only members of a certain group can read it. There is another way to protect a file: encryption. There are a couple times when you should choose to encrypt files over creating a group. Any personal information on your computer like financial records or your social security number should definitely be encrypted. You should also encrypt files if you store files that have business information and company secrets on your computer. In this lab, we'll look at encrypting files using \verb|openssl| or \verb|gnupg|.

\subsection*{Types of Encryptions}

It may be helpful for you to see what types of encryption algorithms you already have on your system. These are the ciphers included on your kernel.

\begin{lstlisting}[basicstyle=\ttfamily, backgroundcolor = \color{lightgray}, language = bash, xleftmargin = 0cm, framexleftmargin = 1em]
cat /proc/crypto | grep name
\end{lstlisting}

The US standard for data encryption is \verb|aes|. \verb|des|, the \verb|Data Encryption Standard|, was the original standard of the Natoinal Institute of Standards and Technology. As time passed though, people began to see the flaws of \verb|des|: namely that \verb|des| used only a 56-bit data encryption key. In 1999, Deep Crack and distributed.net broke a \verb|des| key in 22 hours, revealing how weak \verb|des| encryptions were. Since then, \verb|des| has been replaced by \verb|aes|, the \verb|Advanced Encryption System|. This new algorithm uses a 128-bit block size, but has key sizes that can be either 128, 192, or 256 bits. There are many other encryption algorithms out there.

In the next section, we will be using OpenSSL. Take a look at what encryption algorithms are prebuilt into OpenSSL. 

\begin{lstlisting}[basicstyle=\ttfamily, backgroundcolor = \color{lightgray}, language = bash, xleftmargin = 0cm, framexleftmargin = 1em]
openssl ciphers -v
\end{lstlisting}

\subsection*{OpenSSL}

OpenSSL implements the TLS and SSL protocols and doubles as an encryption library. OpenSSL allows users to securely encrypt files. To ensure that everything functions properly and no errors arise, it's best to have an ingoing and outgoing destination for your encrypted file. The OpenSSL encryption syntax looks like this:

\begin{lstlisting}[basicstyle=\ttfamily, backgroundcolor = \color{lightgray}, language = bash, xleftmargin = 0cm, framexleftmargin = 1em]
openssl [encryption type] -in file-to-encrypt -out encrypted-file
\end{lstlisting}

Try it out for yourself. Create a file \verb|file.txt| and write a secret message inside. Encrypt the file using OpenSSL. 


\begin{lstlisting}[basicstyle=\ttfamily, backgroundcolor = \color{lightgray}, language = bash, xleftmargin = 0cm, framexleftmargin = 1em]
openssl enc des -in file.txt -out encrypted.txt
\end{lstlisting}

This will encrypt your file using the \verb|des| method. You'll be prompted to enter a passphrase. This is the phrase you'll use to decrypt the file, so don't forget it! Once you've done this, \verb|rm file.txt|, the old file, and \verb|less encrypted.txt|. You'll see you can't read your secret now that your file has been encrypted. Looks like you'll have to decrypt the file.  

\begin{lstlisting}[basicstyle=\ttfamily, backgroundcolor = \color{lightgray}, language = bash, xleftmargin = 0cm, framexleftmargin = 1em]
openssl enc -d des -in encrypted.txt -out normal.txt
cat normal.txt
\end{lstlisting}

You can now read your secret! 


\subsection*{gnupg}
\verb|gnupg| or GNU privacy guard, is a tool that can be used to encrypt files. In this case, we are going to use a symmetric-key cipher, which means that both the author and recipient will use the same key for encryption and decryption. Both parties need access to the public key in order to send and receive the message, which is the biggest drawback to the symmetric cipher. To specify a symmetric cipher, add the \verb|-c| option while encrypting. To encrypt a file, run 

\begin{lstlisting}[basicstyle=\ttfamily, backgroundcolor = \color{lightgray}, language = bash, xleftmargin = 0cm, framexleftmargin = 1em]
gpg -c filename
\end{lstlisting}

You will then be asked for a passphrase. After doing this, you'll see that a file, filename.gpg, has been added to the directory. Now that you have the encrypted file, you no longer need the old file so feel free to \verb|rm filename|. If you \verb|cat| the new filename.gpg file, you can see that the information within it is encrypted. 

Unencrypting a file is as easy as encrypting it! 

\begin{lstlisting}[basicstyle=\ttfamily, backgroundcolor = \color{lightgray}, language = bash, xleftmargin = 0cm, framexleftmargin = 1em]
gpg filename.gpg
\end{lstlisting}

There are plenty of other types of encryptions that you can run. Simply look at the man page for more details.

\subsection*{gpg and Mutt}

I bet you've always wanted to send encrypted emails. Well, this is the chance to live your dreams!

It's pretty easy to set up \verb|gpg| on \verb|mutt| once you have your own \verb|gpg| public key, but the process of getting this key can be a bit tricky.

\begin{lstlisting}[basicstyle=\ttfamily, backgroundcolor = \color{lightgray}, language = bash, xleftmargin = 0cm, framexleftmargin = 1em]
gpg --gen-key
\end{lstlisting}

You will then be prompted to to select what kind of key you want. Go ahead and select the default, \verb|RSA and RSA|. Then select 2048 as your keysize and choose how long you want your key to last. After that, you'll need to input some user information. You can use real information or fake--this just makes it easier to locate your key. You'll be asked to enter in a passphrase. Make sure you remember what this password is. If you lose it, you can't get it back! Once you've filled out all the necessary information, GnuPG will generate your key. This might take a while. You can speed the process up by keyboard mashing. \verb|gpg| generates your new key by using the time interval between keystrokes.

\begin{lstlisting}[basicstyle=\ttfamily, backgroundcolor = \color{lightgray}, language = bash, xleftmargin = 0cm, framexleftmargin = 1em]
gpg --list-keys
\end{lstlisting}

You should see your new key within your list of keys. You'll see that there is the \verb|pub|lic key and the \verb|sub|key of the \verb|pub| key. Usually people use the \verb|pub| key for signing and the \verb|sub| key for encryption. 

Before doing anything else, you should create a revocation key. This allows you to revoke your key in the event of a disaster. 

\begin{lstlisting}[basicstyle=\ttfamily, backgroundcolor = \color{lightgray}, language = bash, xleftmargin = 0cm, framexleftmargin = 1em]
gpg --output revoke.asc --gen-revoke <your key>
\end{lstlisting}

Select ``1" and ``yes". 

Now you're ready to integrate \verb|gnupg| into \verb|mutt|. Go into your sample \verb|mutt| configuration files. \verb|mutt| comes pre-installed with a \verb|gnupg| config file within the sample directory.  

\begin{lstlisting}[basicstyle=\ttfamily, backgroundcolor = \color{lightgray}, language = bash, xleftmargin = 0cm, framexleftmargin = 1em]
cd /usr/share/doc/mutt-*/samples
\end{lstlisting}

Inside this directory, you should see a \verb|gpg.rc| file. Copy this file into your home directory so that it'll work for your \verb|mutt|. 

\begin{lstlisting}[basicstyle=\ttfamily, backgroundcolor = \color{lightgray}, language = bash, xleftmargin = 0cm, framexleftmargin = 1em]
cd /usr/share/doc/mutt-*/samples/gpg.rc ~/.gpgrc
\end{lstlisting}

If there is a .bz2 extension after it, you won't be able to copy it to your home directory until you unzip it.

\begin{lstlisting}[basicstyle=\ttfamily, backgroundcolor = \color{lightgray}, language = bash, xleftmargin = 0cm, framexleftmargin = 1em]
bzip2 -d gpg.rc.bz2
\end{lstlisting}

Add the following lines to the \verb|.muttrc| file in your home directory:

\begin{lstlisting}[basicstyle=\ttfamily, backgroundcolor = \color{lightgray}, language = bash, xleftmargin = 0cm, framexleftmargin = 1em]
source = ~./gpgrc
set pgp_use_gpg_agent = yes
set pgp_sign_as = <yourkey> 
set pgp_timeout = 3600
set crypt_replyencrypt = yes
\end{lstlisting}

\subsection*{Finding a Friend}

\indent\indent Look around you and find a buddy to send an email to.

The people you're sending mail to need to have your key or they won't be able to read the mail they receive from you. There are two ways to do this: creating an ASCII version of your key or uploading your key onto a keyserver. 

\begin{enumerate}

\item ASCII Version

\begin{lstlisting}[basicstyle=\ttfamily, backgroundcolor = \color{lightgray}, language = bash, xleftmargin = 0cm, framexleftmargin = 1em]
gpg --armor --output public.key --export <Key ID>
\end{lstlisting}

\verb|public.key| can be whatever name you want. It's helpful to have it be something that's easily identifiable. 

Once that's done, either let your friend know where your key is or send it to them via email. Once you've done that, all they have to do is import it: 

\begin{lstlisting}[basicstyle=\ttfamily, backgroundcolor = \color{lightgray}, language = bash, xleftmargin = 0cm, framexleftmargin = 1em]
gpg --import public.key
\end{lstlisting}

\item Keyservers

Alternatively, to make it easier to find your key, upload your key onto the interwebs.

\begin{lstlisting}[basicstyle=\ttfamily, backgroundcolor = \color{lightgray}, language = bash, xleftmargin = 0cm, framexleftmargin = 1em]
gpg --keyserver subkeys.pgp.net --send-keys <Key ID>
\end{lstlisting}

To have your friend get your key, they just have to do the reverse process, where ``thriller@gmail.com" is the email that you listed in your key's UID. 

\begin{lstlisting}[basicstyle=\ttfamily, backgroundcolor = \color{lightgray}, language = bash, xleftmargin = 0cm, framexleftmargin = 1em]
gpg --keyserver subkeys.pgp.net --search-keys thriller@gmail.com
\end{lstlisting}

Once your friend has found your key, they just have to import it into your keyring. 

\begin{lstlisting}[basicstyle=\ttfamily, backgroundcolor = \color{lightgray}, language = bash, xleftmargin = 0cm, framexleftmargin = 1em]
gpg --import <friend's key ID>
\end{lstlisting}


\end{enumerate}


Once that's done, and you've tried to import your friend's key, check to see if their key is in your keyring. 

\begin{lstlisting}[basicstyle=\ttfamily, backgroundcolor = \color{lightgray}, language = bash, xleftmargin = 0cm, framexleftmargin = 1em]
gpg --list-keys
\end{lstlisting}

If it's there, then everything is perfect! Open mutt and compose a mail to your buddy. Then, on the last screen, instead of pressing ``y" to send your mail, press ``p". You will then get a couple of options for what to do with your mail. Choose ``b"--both sign and encrypt. You will be prompted for your password and for your friend's key. Enter them and everything should go smoothly. Now you're sending encrypted mail!

\subsection*{Activity}

\indent\indent Now that you've configured your mail, grab a partner and try this activity!

One of your close friends, Michael, has recently passed away. On his deathbed, he asked that you retrieve some important data from the file \verb|lastwords.txt| on his account. You've noticed that this file has permissions only for him. Not knowing what to do, you decide you should try using \verb|su| to get into his account. You know that Michael was pretty horrible at choosing passwords. No matter what you told him, he would still choose an English word as his password.

\subsection*{Hint}

To create a password hash file, you have to use \verb|unshadow| to combine the \verb|/etc/passwd| and \verb|/etc/shadow| files.

\begin{lstlisting}[basicstyle=\ttfamily, backgroundcolor = \color{lightgray}, language = bash, xleftmargin = 0cm, framexleftmargin = 1em]
unshadow /etc/passwd /etc/shadow > passfile.txt
sed `/michael/!d' passfile.txt > pass.txt
\end{lstlisting}
Michael is the last entry in the \verb|/etc/passwd| file, so you only need that line to run through john. 

\subsection*{Troubleshooting}

\indent\indent As always, judicious googling is your friend! Here are some other tips just in case things go wrong: 

If you can't get something to work, make sure that you have permissions for that file or directory. 

It can take a while to generate enough entropy to create an GnuPG key. Be patient and if it doesn't work, try making another one or consulting the internet.  

\subsection*{Bonus Reading}
https://xkcd.com/936/

\subsection*{References}

\indent\indent ``GnuPG." Gentoo Wiki. N.p., n.d. Web. 15 July 2014. <http://wiki.gentoo.org/wiki/GnuPG>.

``John The Ripper." Juggernaut AppSec Wiki. N.p., n.d. Web. 15 July 2014. 
<http://juggernaut.wikidot.com/jtr>.

``MuttGuide/UseGPG." Mutt. N.p., Dec. 2013. Web. 15 July 2014. <http://dev.mutt.org/trac/wiki/MuttGuide/UseGPG>.

``SSH/OpenSSH/Keys." Ubuntu Documentation. N.p., n.d. Web. 15 July 2014. <https://help.ubuntu.com/community/SSH/OpenSSH/Keys>.

``Symmetric Key Algorithm." Wikipedia. Wikimedia Foundation, 06 Nov. 2013. Web. 15 July 2014 

``Why AES Is Not Used for Secure Hashing, Instead of SHA-x?" Stack Exchange. N.p., 12 Oct. 2011. Web. 15 July 2014.
<http://security.stackexchange.com/questions/8048/why-aes-is-not-used-for-secure-hashing-instead-of-sha-x>.


````Password" Unseated by ``123456" on SplashData’s Annual ``Worst Passwords" List." SplashData, n.d. Web. 14 July 2014.


Ellingwood, Justin. ``How To Use PAM to Configure Authentication on an Ubuntu 12.04 VPS." DigitalOcean. N.p., 3 Oct. 2013. Web. 14 July 2014.

Ellingwood, Justin. ``How To Use Passwd and Adduser to Manage Passwords on a Linux VPS." DigitalOcean. N.p., 17 June 2014. Web. 15 July 2014. 

Muffett, Alec. ``CrackLib: A ProActive Password Sanity Library." N.p., 14 Dec. 1997. Web. 14 July 2014. <ftp://coast.cs.purdue.edu/pub/tools/unix/libs/cracklib/cracklib.2.7.README>.

NixCraft. ``Linux: HowTo Encrypt And Decrypt Files With A Password." NixCraft. NixCraft, 8 June 2012. Web. 15 July 2014. <http://www.cyberciti.biz/tips/linux-how-to-encrypt-and-decrypt-files-with-a-password.html>.

Pillai, Sarath. ``How Are Passwords Stored in Linux (Understanding Hashing with Shadow Utils)." Slashroot.in. N.p., 24 Apr. 2013. Web. 15 July 2014. <http://www.slashroot.in/how-are-passwords-stored-linux-understanding-hashing-shadow-utils>.

Srivistava, Vishal. ``Understanding and Configuring PAM." DeveloperWorks. IBM, 10 Mar. 2009. Web. 15 July 2014.

Wilson, Gary, Jr. ``Using CrackLib to Require Stronger Passwords." Gary Wilson Jr. N.p., 26 Aug. 2006. Web. 15 July 2014.

\end{document}
