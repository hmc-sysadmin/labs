\documentclass{article}

\usepackage[utf8x]{inputenc}
\usepackage{ucs}
\usepackage{amsmath}
\usepackage{amsfonts}
\usepackage{amssymb}

\usepackage{color}
\usepackage{xcolor}
\usepackage{listings}
\usepackage{url}

\usepackage[margin=1in]{geometry}

\usepackage{titling}
\setlength{\droptitle}{-1in}

\usepackage{courier}

\setlength{\parindent}{0.2in}

\title{System Administration Lab 7: Logs and Backups\vspace{-2ex}}
\author{Paige Rinnert}
\date{}

\begin{document}

\maketitle



\section*{LOGGING}

\indent\indent Many logging servers use syslog, and since almost all machines use it, it is easy to to sync the logging of multiple devices. Once a syslog server is set up, devices can send their log messages over the network to the server rather than recording them in a local file or displaying them. Doing this creates a central logging server that becomes a repository for log messages. \\

A severity level can be used to decide which messages are sent. The levels are standardized and identified by a number and/or a standard abbreviation:

\begin{center}
\begin{tabular}{|c|c|c|}
\hline
Number & Description & Abbr. \\
\hline
\hline
0 & Emergency & emerg \\
\hline
1 & Alerts & alert \\
\hline
2 & Critical & crit \\
\hline
3 & Errors & err \\
\hline
4 & Warnings & warn \\
\hline
5 & Notification & notice \\
\hline
6 & Information & info \\
\hline
7 & Debug & debug \\
\hline
\end{tabular}
\end{center}

There are also things called ``facilities" which loosely relate to system processes, a way of categorizing messages. When a remote device sends a message to a syslog server it includes one of the standard facility values (along with a severity level). Some of the common facilities are:

\begin{center}
\begin{tabular}{|c|c|}
\hline
Code & Explanation \\
\hline
\hline
auth & authentication (login) messages \\
\hline
cron & messages from the memory-resident scheduler \\
\hline
daemon & messages from resident daemons \\
\hline
kern & kernel messages \\
\hline
lpr & printer messages (used by JetDirect cards) \\
\hline
mail & messages from Sendmail \\
\hline
user & messages from user-initiated processes/apps \\
\hline
local0 - local7 & user-defined \\
\hline
syslog & messages from the syslog process itself \\
\hline
\end{tabular}
\end{center}

One can specify different severity levels for different facilities so one can, for example, log all kernel messages but only emergency messages from printers. The logs can then be viewed on the central server. How a server is set up differs with the main operating system being used.


\subsection*{Using sysklogd}

\indent\indent Your system will be running sysklogd. The file used to configure syslog is \verb|/etc/syslog.conf|. In addition to sysklogd, your system is running logrotate. This program, as the name suggests, has the capability for log rotation, which determines how a system saves old logs and creates new ones. \\

If you're interested in the different kinds of loggers that you can use on a Gentoo system, you can read more about them on the Gentoo website: \url{https://www.gentoo.org/doc/en/security/security-handbook.xml?part=1&chap=3}.


\subsection*{Logs found in /var/log}

\indent\indent Run \verb|ls| in the directory \verb|/var/log|. These are the logs maintained by the system. Here are some interesting logs to look at. Usually you need to use sudo to even simply view log files. (Since logs tend to be thousands of lines long, here is a helpful hint for viewing them: in vim, typing SHIFT-G will take you to the bottom of a file while typing gg will return you to the top. Also don't forget that you can search files.)

\begin{enumerate}

\item \verb|auth.log| contains a record of what has been done using the sudo command.

Look at your auth.log file. You should see some lines like this, among others. The line below is a record of using the sudo command to view this file, auth.log, from the directory /var/log (\verb|PWD=/var/log|) using the command vim (\verb|COMMAND=/usr/bin/vim|).

\begin{lstlisting}[basicstyle=\ttfamily, backgroundcolor = \color{lightgray}, language = bash, xleftmargin = 0cm, framexleftmargin = 1em]
Jul  2 10:09:56 penguin sudo: prinnert : TTY=tty1 ; PWD=/var/log ; 
USER=root ; COMMAND=/usr/bin/vim auth.log
\end{lstlisting}

Open your auth.log file and see if you can find some of the commands you have recently performed using sudo.

\item \verb|emerge.log| contains a record of every package emerged on the system, whether the emerge finished successfully or not.

Here is a sample output of a package emerge (this example is emerging telnet):
\begin{lstlisting}[basicstyle=\ttfamily, backgroundcolor = \color{lightgray}, language = bash, xleftmargin = 0cm, framexleftmargin = 1em]
1403904772: Started emerge on: Jun 27, 2014 14:32:52
1403904772:  *** emerge --ask net-misc/telnet-bsd
1403904779:  >>> emerge (1 of 1) net-misc/telnet-bsd-1.2-r1 to /
1403904779:  === (1 of 1) Cleaning (net-misc/telnet-bsd-1.2-r1::/usr/
portage/net-misc/telnet-bsd/telnet-bsd-1.2-r1.ebuild)
1403904780:  === (1 of 1) Compiling/Merging (net-misc/telnet-bsd-1.2-r1::
/usr/portage/net-misc/telnet-bsd/telnet-bsd-1.2-r1.ebuild)
1403904788:  === (1 of 1) Merging (net-misc/telnet-bsd-1.2-r1::/usr/
portage/net-misc/telnet-bsd/telnet-bsd-1.2-r1.ebuild)
1403904789:  >>> AUTOCLEAN: net-misc/telnet-bsd:0
1403904790:  === (1 of 1) Updating world file (net-misc/telnet-bsd-1.2-r1)
1403904790:  === (1 of 1) Post-Build Cleaning (net-misc/telnet-bsd-1.2-r1
::/usr/portage/net-misc/telnet-bsd/telnet-bsd-1.2-r1.ebuild)
1403904790:  ::: completed emerge (1 of 1) net-misc/telnet-bsd-1.2-r1 to /
1403904790:  *** Finished. Cleaning up...
1403904790:  *** exiting successfully.
1403904790:  *** terminating.
\end{lstlisting}

However, below is an example of an entry in the emerge log after attempting to emerge a package that doesn't exist. This is an example of trying to emerge telnet without using the correct package name.

\begin{lstlisting}[basicstyle=\ttfamily, backgroundcolor = \color{lightgray}, language = bash, xleftmargin = 0cm, framexleftmargin = 1em]
1403904498: Started emerge on: Jun 27, 2014 14:28:18
1403904498:  *** emerge  telnet
1403904499:  *** exiting unsuccessfully with status '1'.
1403904499:  *** terminating.
\end{lstlisting}

Note that the program exited unsuccessfully with status 1. A successful command will always exit with status 0 and any other exit status indicates an error. Usually the status 1 is used as a catch-all for all general errors.

Scroll through your emerge.log file and look for some of the packages you've installed recently.

\item \verb|mail.log| contains a record of the emails you have sent and received.

Sample output in mail.log when an email was sent from the computer penguin to the computer heisenberg:
\begin{lstlisting}[basicstyle=\ttfamily, backgroundcolor = \color{lightgray}, language = bash, xleftmargin = 0cm, framexleftmargin = 1em]
Jul  2 10:38:48 penguin postfix/pickup[20871]: C8762CA377D: uid=1000 
from=<root>
Jul  2 10:38:48 penguin postfix/cleanup[21281]: C8762CA377D: message-id=
<20140702173848.GA21269@penguin.sys.cs.hmc.edu>
Jul  2 10:38:48 penguin postfix/qmgr[12656]: C8762CA377D: from=<root
@sys.cs.hmc.edu>, size=533, nrcpt=1 (queue active)
Jul  2 10:38:49 penguin postfix/smtp[21283]: C8762CA377D: to=<root@
heisenberg.sys.cs.hmc.edu>, relay=heisenberg.sys.cs.hmc.edu[134.173.43.69]
:25, delay=0.64, delays=0.17/0.06/0.32/0.08, dsn=2.0.0, status=sent 
(250 2.0.0 Ok: queued as 74E63AE0434)
Jul  2 10:38:49 penguin postfix/qmgr[12656]: C8762CA377D: removed
\end{lstlisting}

Send an email to a friend's computer using either sendmail or mutt, and have them send you an email. Go to your mail log and look at what appears in the log when you send an email and when you receive one.

\item \verb|user.log| contains a record of when your computer has been shut down or rebooted.

\end{enumerate}

Some of the logs update instantaneously (like mail.log when you send or receive an email) while some logs only update when syslogd is restarted or the system is rebooted (like auth.log).


\subsection*{Compressed Files}

\indent\indent Notice that in addition to the green files (the ones ending in .log) and the blue files (directories) in /var/log, there are a bunch of red files. These files all end in the extension .gz, showing that they are zipped (compressed) files. \\

We use compressed files because they take up less space on the system. The programs that zip and unzip files are called \verb|gzip| and \verb|gunzip|, respectively. Try unzipping one of the zipped log files by running the following command. You will probably need to sudo.

\begin{lstlisting}[basicstyle=\ttfamily, backgroundcolor = \color{lightgray}, language = bash, xleftmargin = 0cm, framexleftmargin = 1em]
gunzip <filename>
\end{lstlisting}

Now run \verb|ls| in /var/log. You will see that the file you just unzipped is now gray instead of red and has lost the extension .gz. There is really no reason to have it unzipped though, so go ahead and re-zip it with this command:

\begin{lstlisting}[basicstyle=\ttfamily, backgroundcolor = \color{lightgray}, language = bash, xleftmargin = 0cm, framexleftmargin = 1em]
gzip <filename>
\end{lstlisting}

Now that our zipped log file is back to how it was when we found it, run the following:

\begin{lstlisting}[basicstyle=\ttfamily, backgroundcolor = \color{lightgray}, language = bash, xleftmargin = 0cm, framexleftmargin = 1em]
ls -l | grep auth
\end{lstlisting}

You should see the current auth log file (\verb|auth.log|) as well as any zipped versions of the file. Compare their filesizes. The zipped files may be up to one-tenth the size of the regular, uncompressed log files. Usually compression techniques work by eliminating redundant data or expressing it in a more succinct way. \\

Each of the red, zipped files in /var/log is an old log that is compressed to save space. The logger syslogd automatically compresses them because it makes sense to compress older log files that you most likely won't need. However, it is good to keep these files rather than just deleting them because they may be useful if you have any security breaches or mysterious errors.


\subsection*{Remote Logging}

\indent\indent You may have noticed while perusing your log files that it is very easy to retroactively make changes to a log file or even delete information. It would be entirely possible to delete lines in a mail log so that it looks like a message was never sent, or go into the auth log file and erase a series of commands performed with sudo. One  way to protect your computer against log file alteration by a hacker in covering their tracks is by enabling remote logging. \\

Remote logging allows a computer to log information on another server as well as in its own log files. This means that if information is deleted from the log files on the computer, there is still a record of them  on another computer. These files could be compared to determine if any alterations were made to the log files and what the changes were. \\

Here are the steps to enable remote logging on your system:

\begin{enumerate}

\item Open the file \verb|/etc/syslog.conf|. This file contains lists of the various logs kept by the system. Add the hostname of a friend's computer near the top of the file with this syntax:

\begin{lstlisting}[basicstyle=\ttfamily, backgroundcolor = \color{lightgray}, language = bash, xleftmargin = 0cm, framexleftmargin = 1em]
*.*	@hostname.domainname
\end{lstlisting}

The *.* causes all messages to be logged on the remote host. (If you only wanted one type of message to be logged on the remote host, for example kernel messages, you would replace *.* with kern.*)

\item Since the default behavior of syslogd is that it does not listen on the network, we need to make sure that syslog is set up to listen on the standard port in \verb|/etc/services|. Search for the number 514 and make sure the following line is in the file. If it's not there, add it.

\begin{lstlisting}[basicstyle=\ttfamily, backgroundcolor = \color{lightgray}, language = bash, xleftmargin = 0cm, framexleftmargin = 1em]
syslog	514/udp
\end{lstlisting}

\item Also necessary for syslog to listen to the network is that the -r flag is specified, since the default is that syslog will not receive any messages from the network. You can set this flag on the command line when you start sysklogd (like this:  \verb|syslogd -r|) but this is not permanent and requires you to kill syslogd and then start it again from the command line. A better option is to edit the file \verb|/etc/conf.d/sysklogd| and add the -r flag to the \verb|SYSLOGD| variable. The file should now look like:

\begin{lstlisting}[basicstyle=\ttfamily, backgroundcolor = \color{lightgray}, language = bash, xleftmargin = 0cm, framexleftmargin = 1em]
SYSLOGD="-m 0 -r"
KLOGD="-c 3 -2"
\end{lstlisting}

\item The final step is to run syslogd with the -h flag to enable syslogd to send logs. However, we need to be cautious with this flag because it is easy to create dangerous syslog loops. For this reason, we will not add -h to the SYSLOGD variable in the configuration file, we will simply run it on the command line.

Stop syslogd by running the following (note that the apostrophes are actually backquotes, on the tilde key):

\begin{lstlisting}[basicstyle=\ttfamily, backgroundcolor = \color{lightgray}, language = bash, xleftmargin = 0cm, framexleftmargin = 1em]
kill -SIGTERM `cat /var/run/syslogd.pid`
\end{lstlisting}

Now restart syslogd with the -h flag by running:

\begin{lstlisting}[basicstyle=\ttfamily, backgroundcolor = \color{lightgray}, language = bash, xleftmargin = 0cm, framexleftmargin = 1em]
syslogd -h
\end{lstlisting}

\end{enumerate}

Now you and your partner's computers should contain the same log files! If you are both using remote logging to send messages to each other's computers, the log files on both will contain the same information. If information is deleted out of one of the log files, it will still be present in the other log file. You could then use the diff command to compare log files if a discrepancy arose. This makes your computer safer because if a hacker was trying to cover their tracks, they would have to edit not only the log file on your machine, but also determine that the logs were being saved remotely and delete logs off of the other machine. Possible, but more unlikely. \\

AFTER you have completed everything and checked that everything worked, kill syslog again and restart by simply typing \verb|syslogd| on the command line. We don't want to leave syslogd running with the -h flag for a long time because bad things could happen. \\

\subsubsection*{IMPORTANT!}

Before moving on, run the script \verb|MysteriousScript1| as the superuser. :)

\begin{lstlisting}[basicstyle=\ttfamily, backgroundcolor = \color{lightgray}, language = bash, xleftmargin = 0cm, framexleftmargin = 1em]
sudo ./MysteriousScript1
\end{lstlisting}


\subsubsection*{Extra Credit}

\indent\indent If you want, you can add the -l flag to syslogd to set messages from a remote computer with the same domain to show up in the log files with just the hostname instead of the entire domain. You could specify this either in  the SYSLOGD variable, like this: \verb|SYSLOGD="-m 0 -r -l hostname1:hostname2:..."|; or on the command line when you run syslog, like this: \verb|syslogd -l hostname1:hostname2:...|. \\

Source:  \url{http://www.computerhope.com/unix/sysklogd.htm}

\section*{MONITORING}

\indent\indent It is nice to have logs that you can check for records or if you have problems, but what happens if someone breaks into your computer? It may not be immediately obvious, so it might not be until you go digging through your logs for something else that you realize your system was compromised. \\

The solution? Monitoring! You can set up a service that will send notifications if anything unusual shows up in the logs. The CS department uses Nagios for its monitoring system. If anything suspicious happens, Nagios will send email alerts, allowing people to react quickly and take immediate action if necessary. \\

For example, here is a sample Nagios report in an email from Knuth when the server went down:



\begin{center}
    \parbox{0.5\textwidth}
    {\raggedright
    ***** Nagios *****
    
    Notification Type: PROBLEM
    
    Host: knuth
    
    State: DOWN
    
    Address: knuth.cs.hmc.edu
    
    Info: (Host Check Timed Out)
    
    Date/Time: Thu Jun 5 12:05:16 PDT 2014
    }
\end{center}



Having a server go down is a major problem that needs to be fixed right away, and Nagios alerts allow the system to alert you as soon as something goes wrong. \\

You won't have to install Nagios on your own machines, but just know that it is running on crispy and will notify you if any of your machines go down.



\pagebreak

\section*{BACKUPS}

\indent\indent Backups are an important part of any system, as you may well know if you have ever had your computer crash on you. Saving your files in backups often will save you if anything happens to your system. This is especially important when we are doing so much messing around with our system, because it allows us to restore our files if we screw something up. \\

The CS department uses the program AMANDA for our backups (AMANDA stands for the \textbf{A}dvanced \textbf{M}aryland \textbf{A}utomatic \textbf{N}etwork \textbf{D}isk \textbf{A}rchiver). It was originally developed at the University of Maryland's computer science department and is now a widely used backup system that is completely open source. \\



\subsection*{Making a backup}

\begin{enumerate}

\item The first step now is to create some directories. Create the directory /amanda, and then change its owner to amanda.

\begin{lstlisting}[basicstyle=\ttfamily, backgroundcolor = \color{lightgray}, language = bash, xleftmargin = 0cm, framexleftmargin = 1em]
sudo mkdir -p /amanda
sudo chown amanda /amanda
\end{lstlisting}

Now you will need to switch to the amanda user. Before you can do this, though, you need to change the password for the amanda user. Make sure to write it down somewhere.

\begin{lstlisting}[basicstyle=\ttfamily, backgroundcolor = \color{lightgray}, language = bash, xleftmargin = 0cm, framexleftmargin = 1em]
sudo passwd amanda
\end{lstlisting}

Now you can \verb|su amanda| and create more directories. (Note that the -p flag will create the parent directories if they don't already exist. Also, the curly braces will create multiple directories in the same parent directory).

\begin{lstlisting}[basicstyle=\ttfamily, backgroundcolor = \color{lightgray}, language = bash, xleftmargin = 0cm, framexleftmargin = 1em]
mkdir -p /amanda/vtapes/slot{1,2,3,4}
mkdir -p /amanda/holding
mkdir -p /amanda/state/{curinfo,log,index}
mkdir -p /etc/amanda/MyConfig 
\end{lstlisting}

\item Copy the following config file into \verb|/etc/amanda/MyConfig/amanda.conf|. Notice that the files created in the previous step are mentioned throughout the config file. Each backup will have its own folder and config file. If you want to, you can look at the example amanda.conf file that comes with the amanda install in /etc/amanda/DailySet1.

\begin{lstlisting}[basicstyle=\ttfamily, backgroundcolor = \color{lightgray}, language = bash, xleftmargin = 0cm, framexleftmargin = 1em]
org "MyConfig"
infofile "/amanda/state/curinfo"  # This is the database directory
logdir "/amanda/state/log"        # This is the database for logs
indexdir "/amanda/state/index"    # This is the index directory
dumpuser "amanda"

tpchanger "chg-disk:/amanda/vtapes"  # The device to swap tapes
labelstr "MyData[0-9][0-9]"          # to label the tapes
autolabel "MyData%%" EMPTY VOLUME_ERROR
tapecycle 4    # specifies active volumes that Amanda will not overwrite
dumpcycle 3 days     # each disk gets a full backup at least this often
amrecover_changer "changer"

tapetype "TEST-TAPE"
define tapetype TEST-TAPE {
  length 100 mbytes
  filemark 4 kbytes
}

define dumptype simple-gnutar-local {
    auth "local"
    compress none
    program "GNUTAR"
}

holdingdisk hd1 { #That's a one not an L
    directory "/amanda/holding"
    use 50 mbytes
    chunksize 1 mbyte
} 
\end{lstlisting}

\item In the directory MyConfig, where you just created the config file amanda.conf, create the file \verb|disklist| and put in the line below. All lines in the disklist have the format \verb|<hostname> <diskname> <dumptype>|. The hostname here is localhost and the disk to back up is /home. The dumptype used here is the one that we just created in the config file.

\begin{lstlisting}[basicstyle=\ttfamily, backgroundcolor = \color{lightgray}, language = bash, xleftmargin = 0cm, framexleftmargin = 1em]
localhost /home simple-gnutar-local
\end{lstlisting}

\item Now we need to check that the configuration worked correctly. Since the amanda commands are all located in /usr/sbin, we want to add this to the \verb|PATH| variable for amanda so that we can run the commands without including the path every time. As the user amanda, run \verb|echo $PATH|. If /usr/sbin is not there, add this line to the \verb|.bashrc| file in amanda's home directory (you may need to create the file if it doesn't exist):

\begin{lstlisting}[basicstyle=\ttfamily, backgroundcolor = \color{lightgray}, language = bash, xleftmargin = 0cm, framexleftmargin = 1em]
export PATH=$PATH:/usr/sbin
\end{lstlisting}

To reference this file, now run the following command. Subsequent times, your updated path will be automatically added to the \verb|PATH| variable on restart.

\begin{lstlisting}[basicstyle=\ttfamily, backgroundcolor = \color{lightgray}, language = bash, xleftmargin = 0cm, framexleftmargin = 1em]
source ~/.bashrc
\end{lstlisting}

Now check the path variable again, and you should see /usr/sbin. Now you can run:

\begin{lstlisting}[basicstyle=\ttfamily, backgroundcolor = \color{lightgray}, language = bash, xleftmargin = 0cm, framexleftmargin = 1em]
amcheck MyConfig
\end{lstlisting}

The output should look something like this:

\begin{lstlisting}[basicstyle=\ttfamily, backgroundcolor = \color{lightgray}, language = bash, xleftmargin = 0cm, framexleftmargin = 1em]
Amanda Tape Server Host Check
-----------------------------
NOTE: tapelist will be created on the next run.
Holding disk /amanda/holding: 868352 kB disk space available, using 51200 
kB as requested
slot 1: contains an empty volume
Will write label 'MyData01' to new volume in slot 1.
NOTE: skipping tape-writable test
NOTE: host info dir /amanda/state/curinfo/localhost does not exist
NOTE: it will be created on the next run.
NOTE: index dir /amanda/state/index/localhost does not exist
NOTE: it will be created on the next run.
Server check took 0.276 seconds

Amanda Backup Client Hosts Check
--------------------------------
Client check: 1 host checked in 7.089 seconds.  0 problems found.

(brought to you by Amanda 3.3.3)
\end{lstlisting}

If there are any extra lines in your output that look like they might be errors, you will need to resolve these before moving on.

\item Now you can run your first backup!

\begin{lstlisting}[basicstyle=\ttfamily, backgroundcolor = \color{lightgray}, language = bash, xleftmargin = 0cm, framexleftmargin = 1em]
amdump MyConfig
\end{lstlisting}

The command should take a few seconds to run, that's probably a good thing. When it's finished, try \verb|echo $?|. If the result is anything other than 0, it means that amdump ended in an error state and did not run correctly. If it is 0, you're golden.

\item Now run this command:

\begin{lstlisting}[basicstyle=\ttfamily, backgroundcolor = \color{lightgray}, language = bash, xleftmargin = 0cm, framexleftmargin = 1em]
amreport MyConfig > /amanda/amandareport
\end{lstlisting}

This command puts the report from running the backup into a file in the amanda directory. Go view the file you just created. This is the output that gets emailed to CS staff whenever amanda runs a backup on Diffie.

\end{enumerate}

Congratulations, you've finished your first backup! Now let's figure out how to restore the files that have been backed up.


\subsection*{Restoring a backup}

\begin{enumerate}

\item There are a few changes you need to make to the amanda client configuration file, \\* \verb|/etc/amanda/amanda-client.conf|. Change the lines in this file to match the ones below:

\begin{lstlisting}[basicstyle=\ttfamily, backgroundcolor = \color{lightgray}, language = bash, xleftmargin = 0cm, framexleftmargin = 1em]
conf "MyConfig"
index_server "localhost"
tape_server "localhost"
tapedev "changer"
auth "local"
\end{lstlisting}

Don't worry about the \verb|ssh_keys| line, just leave it as the empty string.

\item Now we need to create a place where the backup will restore to. Go to \verb|/tmp| and create the directory \verb|test-recovery|. Go to that directory.

\begin{lstlisting}[basicstyle=\ttfamily, backgroundcolor = \color{lightgray}, language = bash, xleftmargin = 0cm, framexleftmargin = 1em]
mkdir /tmp/test-recovery
cd /tmp/test-recovery
\end{lstlisting}

\item Now you're ready to restore the files. Since you need permission to access files on the system, \verb|amrecover| needs to be run with sudo.

\begin{lstlisting}[basicstyle=\ttfamily, backgroundcolor = \color{lightgray}, language = bash, xleftmargin = 0cm, framexleftmargin = 1em]
amrecover MyConfig
\end{lstlisting}

You should see something similar to the following. Now amrecover will prompt you to enter information.

\begin{lstlisting}[basicstyle=\ttfamily, backgroundcolor = \color{lightgray}, language = bash, xleftmargin = 0cm, framexleftmargin = 1em]
Using index server from environment AMANDA_SERVER (penguin)
AMRECOVER Version 3.3.3. Contacting server on penguin ...
220 penguin AMANDA index server (3.3.3) ready.
Setting restore date to today (2014-07-08)
200 Working date set to 2014-07-08.
200 Config set to MyConfig.
501 Host penguin is not in your disklist.
Trying host localhost ...
200 Dump host set to localhost.
Use the setdisk command to choose dump disk to recover
amrecover>
\end{lstlisting}

Lines with errors begin with a 5 while the lines that work correctly begin with a 2. If the program complains that the host is not in your disklist, set the host to localhost by typing the following. Otherwise, keep going.

\begin{lstlisting}[basicstyle=\ttfamily, backgroundcolor = \color{lightgray}, language = bash, xleftmargin = 0cm, framexleftmargin = 1em]
amrecover> sethost localhost
\end{lstlisting}

\item Now set the disk to /home, the directory you backed up.

\begin{lstlisting}[basicstyle=\ttfamily, backgroundcolor = \color{lightgray}, language = bash, xleftmargin = 0cm, framexleftmargin = 1em]
amrecover> setdisk /home
\end{lstlisting}

Amrecover should confirm that it did the proposed action by printing the following:

\begin{lstlisting}[basicstyle=\ttfamily, backgroundcolor = \color{lightgray}, language = bash, xleftmargin = 0cm, framexleftmargin = 1em]
200 Disk set to /home.
\end{lstlisting}

\item And add the name of your home directory (should be your user name):

\begin{lstlisting}[basicstyle=\ttfamily, backgroundcolor = \color{lightgray}, language = bash, xleftmargin = 0cm, framexleftmargin = 1em]
amrecover> add <directory name>
\end{lstlisting}

Again, amrecover should confirm that it added the directory:

\begin{lstlisting}[basicstyle=\ttfamily, backgroundcolor = \color{lightgray}, language = bash, xleftmargin = 0cm, framexleftmargin = 1em]
Added directory <directory name>
\end{lstlisting}

\item Now you're ready to extract files.

\begin{lstlisting}[basicstyle=\ttfamily, backgroundcolor = \color{lightgray}, language = bash, xleftmargin = 0cm, framexleftmargin = 1em]
amrecover> extract
\end{lstlisting}

Amrecover will ask you to confirm what you want to do, and if it looks right, type Y (make sure it is a capital Y or it will get angry with you).

\begin{lstlisting}[basicstyle=\ttfamily, backgroundcolor = \color{lightgray}, language = bash, xleftmargin = 0cm, framexleftmargin = 1em]
Extracting files using tape drive changer on host localhost.
The following tapes are needed: MyData01

Extracting files using tape drive changer on host localhost.
Load tape MyData01 now
Continue [?/Y/n/s/d]? Y
\end{lstlisting}

Amrecover should list the recovered files, returning you to the amrecover prompt. If so, you have successfully restored your home directory! That is the end of the recovery process, so type CTRL-D to exit amrecover and return to your command prompt.

\begin{lstlisting}[basicstyle=\ttfamily, backgroundcolor = \color{lightgray}, language = bash, xleftmargin = 0cm, framexleftmargin = 1em]
amrecover>
\end{lstlisting}

If you run \verb|ls| in your current directory, \verb|/tmp/test-recovery|, you should see a copy of your home directory.

\end{enumerate}


\subsection*{ACTIVITY!}

\indent\indent Now that you've successfully backed up your home directory, let's back up more files using the same process. Add a new configuration called MyLogs to back up the disk \verb|/var/log|. To start out, as the amanda user copy the MyConfig directory and name the new directory MyLogs. You will need to change the name of the configuration in \verb|/etc/amanda/MyLogs/amanda.conf| and \verb|/etc/amanda/amanda-client.conf|, and change the disk to back up in \verb|/etc/amanda/MyLogs/disklist|. Then run \verb|amcheck MyLogs| to check the new configuration and \verb|amdump MyLogs| to actually perform the backup.


\subsubsection*{Resources}

\indent\indent \url{http://wiki.zmanda.com/index.php/GSWA/Build_a_Basic_Configuration}

\url{http://wiki.zmanda.com/index.php/GSWA/Recovering_Files}


\subsubsection*{IMPORTANT!}

Before continuing, run the script \verb|MysteriousScript2| as the superuser.

\begin{lstlisting}[basicstyle=\ttfamily, backgroundcolor = \color{lightgray}, language = bash, xleftmargin = 0cm, framexleftmargin = 1em]
sudo ./MysteriousScript2
\end{lstlisting}



\section*{CRON JOBS}

\indent\indent Backups are so important that you should do them often, at regular intervals. It can be difficult to remember to do backups, so that is one of the reasons that cron jobs exist. Cron jobs automate processes so they will run at set intervals, which is perfect for keeping a system's backups up-to-date. \\

The cron jobs running on your system are listed in the \verb|/etc/crontab|. Go take a look at the file. Here is a sample line from this file:

\begin{lstlisting}[basicstyle=\ttfamily, backgroundcolor = \color{lightgray}, language = bash, xleftmargin = 0cm, framexleftmargin = 1em]
19 4 * * 0,3,6  root	rm -f /var/spool/cron/lastrun/cron.weekly
\end{lstlisting}

The last part of the line is fairly straightforward. \verb|root| is the user running the command, and following root is the command itself. The beginning of the line may look ambiguous, but it describes how often the job should run. The numbers represent:

\begin{lstlisting}[basicstyle=\ttfamily, backgroundcolor = \color{lightgray}, language = bash, xleftmargin = 0cm, framexleftmargin = 1em]
19        4        *                *        0,3,6
minutes   hours    day (of month)   month    day (of week)
(0-59)    (0-23)   (1-31)           (1-12)   (0-6)         # allowable values
\end{lstlisting}

So, this particular command removes the file \verb|cron.weekly| at 4:19am every Sunday, Wednesday, and Saturday. \\

Add two lines to the crontab file to check the configuration of MyConfig at 9pm and run a backup every day at 9:30pm. Set up amcheck so that instead of printing to standard output like it usually does, it mails the output to root. (Reminder: the amanda commands have man pages just like any other command!)


\subsubsection*{Check your computer's email}

\indent\indent Anything interesting in the root account?

\end{document}




