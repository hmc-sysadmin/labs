\documentclass{article}

\usepackage[utf8x]{inputenc}
\usepackage{ucs}
\usepackage{amsmath}
\usepackage{amsfonts}
\usepackage{amssymb}

\usepackage{color}
\usepackage{xcolor}
\usepackage{listings}
\usepackage{url}

\usepackage[margin=1in]{geometry}

\usepackage{titling}
\setlength{\droptitle}{-1in}

\usepackage{courier}

\setlength{\parindent}{0.2in}

\title{System Administration Lab 3: Shell Scripting in Bash\vspace{-2ex}}
\author{Lisa Goeller, Jozi McKiernan, and Paige Rinnert}
\date{}

\begin{document}

\maketitle

\section*{Readings}
Before beginning this lab, here are some readings you should do. They come from the book \textit{The Linux Command Line: A Complete Introduction} by William Shotts. \\

\textbf{Recommended reading:} Chapters 24-25 (p. 309-324) provide some background on setting and using variables.

\textbf{Required reading:} Chapters 26-29 (p. 325-362) discuss functions, if/else branching, keyboard input, and while/until loops.

\textbf{Extra reading:} Chapter 31 (p. 375-380) talks about case branching, which will be useful for the extra credit.


\section*{Introduction}

\indent\indent The goal of this lab is to write a program in bash to play Hangman! PAIR PROGRAMMING IS HIGHLY ENCOURAGED FOR THIS LAB :) Bash is a scripting language. There are different kinds of shells (like sh, zsh, and csh) but we will just be working in bash, which stands for \textbf{b}ourne \textbf{a}gain \textbf{sh}ell. It is a replacement for sh, the original Bourne shell. The scripting language is defined at the top of a program with the ``shebang" character: \verb|#!|. Every program starts with a shebang! ;) \\

We hope you read the background information on shell scripting. To do this lab, you need to be familiar with constructing if/else statements and while/until loops. You should also be fairly comfortable with regular expressions. Don't be afraid to use Google to look up syntax that you are unsure about! This is a helpful online resource: \url{http://www.johnstowers.co.nz/blog/pages/bash-cheat-sheet.html} \\

The file you will be editing is \verb|hangman_lab|. Be sure to copy the file to your home directory. Also copy the file \verb|dict| to your home directory. This is the file you will grab words from; each word is on a new line in the file. Go ahead and use less to view the dictionary file if you want.



\section*{Running a script}

\indent\indent Before you start writing, here is how you run a script. You can run a script from the command line if it is in your path. The path is set by the \verb|$PATH| variable in your environment. If the script is in a directory on your path, you will be able to run it by typing the name of the script on the command line. \\

To find out what directories are part of your path, type \verb|echo $PATH| on your command line. It  will produce a colon-delimited list of directories on your path. Be sure to copy your hangman script to a directory on your path, preferably one tied to your user account, like \verb|/usr/local/bin|. \\

Throughout the process of writing the script, you can check periodically that it is working by running it and seeing if the behavior is what you expected. Be sure that any incomplete lines of code are commented out so they don't return errors. Feel free to add print statements throughout to help with debugging.



\section*{Defining Variables}

\indent\indent Start off by completing the function \verb|define| that defines the necessary variables. We have defined three of them for you: \verb|word|, \verb|length|, and \verb|wip|. The \verb|word| is selected randomly from the provided dictionary using the command \verb|shuf|, which shuffles the dictionary and returns the first line (a single word). (NOTE: You need to add the pathway to the file \verb|dict|!)The length of the word needs to be defined so that it can be referenced in \verb|wip| (the word in progress). The word in progress begins as simply underscores representing each letter of the word, and this will become the ``board" for the game. \\

The variables you need to add are \verb|lives| and \verb|gl|. \verb|lives| will be the number of lives the player gets before they lose the game. In our example we gave 8 lives to the player. \verb|gl| is an empty array that will eventually contain the letters that the player has guessed. An array, which is similar to a list, is represented by parentheses: \verb|()|. A note about setting variables is that bash cares about whitespace. There shouldn't be any space between the variable and its value, like this: \verb|variable=value|. \\

Remember that to refer to variables in bash, the syntax is \verb|$variable|. If you want to include a character directly after the variable (for example a period at the end of a sentence), include curly brackets around the variable: \verb|${variable}|. To reference arrays, in addition to the \$ and curly brackets, you need to tell the shell to look at all elements of the array: \verb|${array[@]}|. If you are setting the results of a command to a variable, like in \verb|word|, \verb|length|, and \verb|wip|, the command needs to be enclosed in parentheses and then preceded by a \$, like this: \verb|variable=$(command)|.



\section*{Get user input for letter}

\indent\indent The function you will write next is called \verb|get_letter|. (Instead of writing the functions in the order they will be called, which is the order they appear in the file, we encourage you to write them in the order they appear in this handout.) The main job of this function is to ask the user to input a letter. There are three parts to the function: first you print a string to tell the user what to enter, next comes the command to read the user's input, and finally the function calls \verb|valid_letter| to determine if the input was valid. \\

Remember the print command in bash is \verb|echo|. The command to ask for user input is \verb|read| and the syntax is \verb|read <variable>|. In this case, we will call the variable \verb|letter|. After adding code to a function, it's okay to remove the \verb|return| statements.



\section*{Check that letter is valid}

\indent\indent The next function you should write is called \verb|valid_letter| and (surprisingly) checks whether the user input for \verb|letter| is valid. 
Valid inputs consist of lowercase Roman alphabet characters only. If the input is valid, the script should call the function \verb|guessed_letter|. (We have written this function for you because the syntax is kind of obscure.) After the \verb|if| condition include a semicolon followed by the word \verb|then|. \\ 

If the input is invalid, there should be an \verb|else| statement where the script prompts for another letter by calling \verb|get_letter|. You might consider printing some additional helpful information for your user, such as a reminder of what letters have been guessed and what the word looks like now. At the end of the \verb|else| statement, you need to type \verb|fi| to close the entire if/else section.



\section*{Check if the letter has been guessed}

\indent\indent We wrote this function, \verb|guessed_letter|, for you because the syntax is bizarre. The goal of this function is to check if the user has already guessed the letter. If the user has guessed it, the function returns to \verb|get_letter| to ask for a different letter. If not, the function adds the letter to the array \verb|gl| which contains  guessed letters, and it calls \verb|letter_in_word| to determine whether the letter is in the word.



\section*{Check that the letter is in the word}

\indent\indent This function, \verb|letter_in_word|, checks whether the letter is in the given word. There should be an \verb|if| case for if the letter is in the word, and an \verb|else| case for if it is not. \\

For the \verb|if| case, use regular expressions to check whether the string \verb|$word| contains the string \verb|$letter|. Inside this \verb|if| statement, update the variable \verb|$wip| to include the newly guessed letter. To do this, we will use a find and replace command within a string. We want to replace all characters in \verb|$word| NOT found in the guessed letters list with underscores. This is slightly counterinuitive but it works. The syntax of the find and replace command for strings in bash is \verb|${string//pattern/replacement}|. Use regex to specify the pattern. Don't forget to include any text you want to display, like the newly updated board and the letters that have been guessed. \\

For the \verb|else| case, we don't need to make any changes to \verb|$wip| because the letter isn't in the word. However, we do need to decrement the number of lives. You will need to use double parentheses (( )) to create a math environment, with a \$ in front of them to mark it as a variable. You also need to call \verb|check_lives| to determine whether the user still has lives left or if the game is over. (You will write this function later.) Don't forget \verb|fi|.



\section*{Write game function}

\indent\indent You will now need to write a function \verb|game| that prompts the user for letters until the word is guessed. 
In order to have an effective game, you first need to define your variables (funny, I think you wrote a function for that already). 
You also need to let the player know how long the word is by printing the board. 
Then, you need to keep asking for letters until the word-in-progress is the same as the word itself. (Remember, an \verb|until| loop requires \verb|do| at the beginning and \verb|done| when it is finished.)
When the word is guessed, the user should be asked if they want to play again.
Their \verb|answer| should be read in. 



\section*{Check number of lives remaining}

\indent\indent Now you will need to check to see if the player has any lives remaining. You can do this by writing an \verb|if| statement.
If the player has no lives remaining, you should give them \verb|$word| (just to be nice) and the option to play again if they so wish. Otherwise, the player should just keep playing. If the player decides to play again, both the word and the variables need to be reset.



\section*{Putting it all together}

\indent\indent Now that you've defined the functions, you should call them, assuming that the user wants to play. (Hint: Try using a while loop!) Consider how all the functions can work together to allow the game to actually work.

Now you're ready to run the script and test your game!



\section*{Extra Credit :)}

\indent\indent If you still have some time left over, you can write a starting menu that offers the option for the player to play in single-player mode or against a friend. To write a menu, look at \verb|case| branching (see Chapter 31 of the Linux book). It is within these branches that you should define \verb|$word|, whether it should be retrieved randomly from a dictionary or whether it should be inputted into the game. \\

Also, if you are so inclined, you can try including ASCII art of a hangman (or something less grotesque) that updates throughout the game. Another idea is to create a cheating function that allows the player to search through the dictionary with regex for possible matches to the letters they've already guessed in the word. Play around, have fun, and happy scripting!




\end{document}