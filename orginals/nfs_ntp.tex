\documentclass[11pt]{article}
\usepackage[utf8x]{inputenc}
\usepackage{ucs}
\usepackage{amsmath}
\usepackage{amsfonts}
\usepackage{amssymb}
\usepackage{hyperref}
\usepackage{titling}
\usepackage{color}
\usepackage{xcolor}
\usepackage{listings}

\usepackage[margin=1in]{geometry}

\setlength{\droptitle}{-1in}

\usepackage{courier}
\setlength{\parindent}{0.2in}

\title{System Administration Lab 4: NTP and NFS\vspace{-2ex}}
\author{Annie Christy}
\date{}

\begin{document}

\maketitle

Note: Most of the commands in this lab need to be run as the superuser, so consider running them as root.

\section*{Network Time Protocol (NTP)}

\subsection*{What is NTP?}

\indent\indent NTP is an internet protocol that uses packet-switched, variable-latency data networks to synchronize system clocks. The purpose of NTP is to synchronize the clocks of all connected computers within a few milliseconds of Coordinated Universal Time, which is also known as UTC. \\
		
		NTP clocks are organized into strata, where stratum 0 is made up of extremely precise atomic clocks, GPS clocks, or radio clocks. Stratum 1 is made up of computers that are coordinated with the atomic devices. The lower strata (2 through 15) are a tree of synchronized servers that communicate with each other to achieve synchronization. \\
		
	   We are interested in this protocol because some of the services we want to run depend on servers and clients that have exactly synchronized time. Specifically, NFS will complain if the client and server clocks are not synchronized. \\
	
	
For more information, visit 
\url{http://en.wikipedia.org/wiki/Network_Time_Protocol}
	
	\subsection*{Enabling NTP}
	
\indent\indent The class server is already running NTP. Your goal is to use NTP to sync your system clock to the server's clock. 

	You want to tell your computer to query NFS server for NTP synchronization. To do this, open the /etc/ntp.conf file and add a line directing the NTP implementation to the class server.
	
	\begin{lstlisting}[basicstyle=\ttfamily, backgroundcolor = \color{lightgray}, language = bash, xleftmargin = 0cm, framexleftmargin = 1em]
sudo vim /etc/ntp.conf
\end{lstlisting}

Add the following lines to the section whose comments tell you to ``Name of the servers ntpd should sync with."

\begin{lstlisting}[basicstyle=\ttfamily, backgroundcolor = \color{lightgray}, language = bash, xleftmargin = 0cm, framexleftmargin = 1em]
server          crispy.sys.cs.hmc.edu
server          127.127.1.0
fudge           127.127.1.0          stratum 10
\end{lstlisting}

The first line instructs ntpd to sync with the class NFS server. The following two lines tell ntpd to synchronize with itself if it loses its connection with the class server.

Note the presence of a driftfile in \verb|/etc/ntp.conf|. This file is where the machine can store information about the natural time drifts that your computer has, so that it can automatically adjust for them.

You need to uncomment the bottom line in the file to ensure that other machines on your network can synchronize with your server, but that they cannot configure the server.

\begin{lstlisting}[basicstyle=\ttfamily, backgroundcolor = \color{lightgray}, language = bash, xleftmargin = 0cm, framexleftmargin = 1em]
restrict 192.168.0.0 mask 255.255.255.0 nomodify nopeer notrap
\end{lstlisting}

the NTP service has a function called monlist, that is used to query the server about the hosts that have connected to it. This can be useful if you want to learn about your network stats, but we don't really need it and it leaves your machine susceptible to denial-of-service attacks. To disable this functionality, add the following line to the bottom of the /etc/ntp.conf file:

\begin{lstlisting}[basicstyle=\ttfamily, backgroundcolor = \color{lightgray}, language = bash, xleftmargin = 0cm, framexleftmargin = 1em]
disable monitor
\end{lstlisting} 

To start the NTP service and check that it is running, run the following rc-service commands as root.

\begin{lstlisting}[basicstyle=\ttfamily, backgroundcolor = \color{lightgray}, language = bash, xleftmargin = 0cm, framexleftmargin = 1em]
rc-service ntp-client start
\end{lstlisting} 

\begin{lstlisting}[basicstyle=\ttfamily, backgroundcolor = \color{lightgray}, language = bash, xleftmargin = 0cm, framexleftmargin = 1em]
rc-service ntp-client status
\end{lstlisting} 

Add the service to your default run level so that NTP will start at boot.
\begin{lstlisting}[basicstyle=\ttfamily, backgroundcolor = \color{lightgray}, language = bash, xleftmargin = 0cm, framexleftmargin = 1em]
rc-update add ntp-client default
\end{lstlisting} 

To start ntpd, run the following line.
\begin{lstlisting}[basicstyle=\ttfamily, backgroundcolor = \color{lightgray}, language = bash, xleftmargin = 0cm, framexleftmargin = 1em]
/etc/init.d/ntpd start
\end{lstlisting} 


%Then, run the following line, and add ntpd to the default run level.
%\begin{lstlisting}[basicstyle=\ttfamily, backgroundcolor = \color{lightgray}, language = bash, xleftmargin = 0cm, %framexleftmargin = 1em]
%rc-service ntpd start
%\end{lstlisting} 

Then, add ntpd to your default run level.

\begin{lstlisting}[basicstyle=\ttfamily, backgroundcolor = \color{lightgray}, language = bash, xleftmargin = 0cm, framexleftmargin = 1em]
rc-update add ntpd default
\end{lstlisting} 



If you would like to monitor the status of your server.
\begin{lstlisting}[basicstyle=\ttfamily, backgroundcolor = \color{lightgray}, language = bash, xleftmargin = 0cm, framexleftmargin = 1em]
rc-status
\end{lstlisting} 


%NOTE FOR EDITS: THERE ARE MORE STEPS ON THE WIKI, BUT I THINK THEY ARE FOR SETTING UP A SERVER AND ALL WE %REALLY NEED ARE CLIENTS. MOSTLY ABOUT HARDWARE CLOCKS.

For more information, visit 
\url{http://wiki.gentoo.org/wiki/Ntp}



\section*{Network File System (NFS)}

\subsection*{What is NFS?}

\indent\indent In 1984, Sun Microsystems developed a distributed file system protocol called a Network File System (NFS). NFS makes it possible for clients to access files on a remote server as though they were a part of the local file tree. The class server is already running the NFS service. It is running the NFS daemon, nfsd, and it will reference the file /etc/exports to determine what filesystems it is allowed to export to clients. I have already added your computers to this file and given them read-write access to the /local directory. Your first goal will be to mount /local to your machine.\\
		
		To access the NFS shares on the server, we will use tools from the nfs-utils package, which has already been installed.	
		
%NOTE!!! I'M NOT SURE THAT YOU ACTUALLY NEED TO EMERGE RPCBIND. IT MAY COME WITH THE PREVIOUS EMERGE. %SOMEONE SHOULD TRY A FEW MORE STEPS WITHOUT IT AND LET ME KNOW :O
%\begin{lstlisting}[basicstyle=\ttfamily, backgroundcolor = \color{lightgray}, language = bash, xleftmargin = 0cm, %framexleftmargin = 1em]
%emerge -a rpcbind
%\end{lstlisting} 	

Start the rpc.statd daemon.

\begin{lstlisting}[basicstyle=\ttfamily, backgroundcolor = \color{lightgray}, language = bash, xleftmargin = 0cm, framexleftmargin = 1em]
/etc/init.d/rpc.statd start
\end{lstlisting} 	


	For more information, visit 

	\url{http://en.wikipedia.org/wiki/Network_File_System}
	
	\subsection*{Manually Mounting NFS Shares}
		We will begin by learning how to manually mount the desired filesystems onto your machine. 
		
		You need to create the directory that you want the NFS share to mount to. It doesn't really matter what you call this directory, but it makes sense to call it by the same name that it has on the server. So, create a file called /local.

\begin{lstlisting}[basicstyle=\ttfamily, backgroundcolor = \color{lightgray}, language = bash, xleftmargin = 0cm, framexleftmargin = 1em]
cd /
mkdir /local
\end{lstlisting} 	

Just to convince you that the filesystem is not mounted yet, please go into /local. It should be empty.
\begin{lstlisting}[basicstyle=\ttfamily, backgroundcolor = \color{lightgray}, language = bash, xleftmargin = 0cm, framexleftmargin = 1em]
cd /local
cd ..
\end{lstlisting} 	

Now, mount the filesystem from the server.This command takes the form of mount server.address:/server/path /client/path.
\begin{lstlisting}[basicstyle=\ttfamily, backgroundcolor = \color{lightgray}, language = bash, xleftmargin = 0cm, framexleftmargin = 1em]
mount crispy.sys.cs.hmc.edu:/vol/local /local
\end{lstlisting} 	

Now, if you go into the /local file, you should see several folders. Open the file called README and follow the instructions there.

It is good to know how to manually mount filesystems, but it is not a particularly efficient way to work if you want to use the NFS shares a lot because the share becomes unmounted when the computer is rebooted. To prove this to yourself, you should reboot your computer now (but first make sure that you don't have any important processes running). 

\begin{lstlisting}[basicstyle=\ttfamily, backgroundcolor = \color{lightgray}, language = bash, xleftmargin = 0cm, framexleftmargin = 1em]
sudo shutdown -r now
\end{lstlisting} 	

Now, go back to the /local directory. It should be empty because your machine isn't getting the files off the NFS server anymore. You don't need this directory anymore, so you can delete it.

\begin{lstlisting}[basicstyle=\ttfamily, backgroundcolor = \color{lightgray}, language = bash, xleftmargin = 0cm, framexleftmargin = 1em]
rmdir /local
\end{lstlisting} 	
	
	\subsection*{Mount NFS Shares at Boot with autofs}

\indent\indent You may have noticed that if you change something in your home directory on knuth, that change will appear when you log into your home directory on any of the other machines on the CS Department cluster. This works because the home directories are exported to the machines using NFS. However, it is unlikely that you have ever had to mount your home directory on knuth before working with it. That would be really annoying because you would need to do it every time you wanted to access your files! Instead, it is possible to mount the NFS shares automatically when then computer boots. That way, the NFS files that you use a lot will be waiting for you when you log on.
	
	We use a program called autofs to automount files. 	
There are two files that we need to configure on the client machine. First, read through the file /etc/autofs/auto.master.

\begin{lstlisting}[basicstyle=\ttfamily, backgroundcolor = \color{lightgray}, language = bash, xleftmargin = 0cm, framexleftmargin = 1em]
sudo vim /etc/autofs/auto.master
\end{lstlisting} 	

This file tells autofs where to mount files and also provides a path to a file that provides details about what to mount.
Add the following line to the file beneath the comment that says  'For details of the format look at autofs(5)' but make sure the line you add is uncommented.
\begin{lstlisting}[basicstyle=\ttfamily, backgroundcolor = \color{lightgray}, language = bash, xleftmargin = 0cm, framexleftmargin = 1em]
/mnt		/etc/autofs/auto_mnt
\end{lstlisting} 	

The line you just added to the file tells the computer to mount the NFS shares in the /mnt directory. It also gives the computer the path to the auto\textunderscore mnt file which will contain a list of the NFS shares that should be mounted automatically. Open that file now.

\begin{lstlisting}[basicstyle=\ttfamily, backgroundcolor = \color{lightgray}, language = bash, xleftmargin = 0cm, framexleftmargin = 1em]
sudo vim /etc/autofs/auto_mnt
\end{lstlisting} 	

This file should be empty when you first open it. Each line you add will represent a directory that you want to mount to your system. The first part of a line is the name of the share as you want it to show up in your /mnt directory. the second part of a line provides the location of the share on the NFS server. Tell the machine to automount /vol/local by adding the following line.

\begin{lstlisting}[basicstyle=\ttfamily, backgroundcolor = \color{lightgray}, language = bash, xleftmargin = 0cm, framexleftmargin = 1em]
local crispy.sys.cs.hmc.edu:/vol/local
\end{lstlisting} 	

If you want autofs to mount the NFS share at boot, then you need to make sure that autofs itself is running at boot. Make sure that ntpd, nfs, and autofs are all added to the default run level.
\begin{lstlisting}[basicstyle=\ttfamily, backgroundcolor = \color{lightgray}, language = bash, xleftmargin = 0cm, framexleftmargin = 1em]
rc-update add nfs default
\end{lstlisting} 	

\begin{lstlisting}[basicstyle=\ttfamily, backgroundcolor = \color{lightgray}, language = bash, xleftmargin = 0cm, framexleftmargin = 1em]
rc-update add autofs default
\end{lstlisting} 	

The ntpd service should already be on the default run level, but you can use the rc-status command to check. It will print a list of all the services that start at boot.
\begin{lstlisting}[basicstyle=\ttfamily, backgroundcolor = \color{lightgray}, language = bash, xleftmargin = 0cm, framexleftmargin = 1em]
rc-status
\end{lstlisting} 	

Now, reboot your machine.
\begin{lstlisting}[basicstyle=\ttfamily, backgroundcolor = \color{lightgray}, language = bash, xleftmargin = 0cm, framexleftmargin = 1em]
sudo shutdown -r now
\end{lstlisting} 	

The files should be mounted automatically. Check to make sure.
\begin{lstlisting}[basicstyle=\ttfamily, backgroundcolor = \color{lightgray}, language = bash, xleftmargin = 0cm, framexleftmargin = 1em]
cd /mnt/local
\end{lstlisting} 	

In the directory you created earlier, which should be located at \\* \verb|/mnt/local/classDir/firstnameLastname|, answer the following questions.\\

1. What was the most difficult part of this lab?\\

2. What did you do to troubleshoot your problems?\\

3. Explain NFS to a twelve-year-old.\\




\end{document}